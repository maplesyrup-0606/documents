%% Standard start of a latex document
\documentclass[letterpaper,12pt]{article}
%% Always use 12pt - it is much easier to read
%% Things written after '%' sign, are ignored by the latex editor - they are how to introduce comments into your .tex source
%% Anything mathematics related should be put in between '$' signs.

%% Set some names and numbers here so we can use them below
\newcommand{\myname}{Mercury Mcindoe} %%%%%%%%%%%%%%% ---------> Change this to your name
\newcommand{\mynumber}{85594505} %%%%%%%%%%%%%%% ---------> Change this to your student number
\newcommand{\hw}{11} %%%%%%%%%%%%%%% --------->  set this to the homework number

%%%%%%
%% There is a bit of stuff below which you should not have to change
%%%%%%

%% AMS mathematics packages - they contain many useful fonts and symbols.
\usepackage{amsmath, amsfonts, amssymb, amsthm}

%% The geometry package changes the margins to use more of the page, I suggest
%% using it because standard latex margins are chosen for articles and letters,
%% not homework.
\usepackage[paper=letterpaper,left=25mm,right=25mm,top=3cm,bottom=25mm]{geometry}
%% For details of how this package work, google the ``latex geometry documentation''.

%%
%% Fancy headers and footers - make the document look nice
\usepackage{fancyhdr} %% for details on how this work, search-engine ``fancyhdr documentation''
\pagestyle{fancy}
%%
%% The header
\lhead{Mathematics 220} % course name as top-left
\chead{Homework \hw} % homework number in top-centre
\rhead{ \myname \\ \mynumber }
%% This is a little more complicated because we have used `` \\ '' to force a line-break between the name and number.
%%
%% The footer
\lfoot{\myname} % name on bottom-left
\cfoot{Page \thepage} % page in middle
\rfoot{\mynumber} % student number on bottom-right
%%
%% These put horizontal lines between the main text and header and footer.
\renewcommand{\headrulewidth}{0.4pt}
\renewcommand{\footrulewidth}{0.4pt}
%%%

%%%%%%
%% We shouldn't have to change the stuff above, but if you want to add some newcommands and things like that, then putting them between here and the ``\begin{document}'' is a good idea.
%%%%%%
%% A useful command to define is
%% This command will make the left and right braces as tall as needed. Use it as \set{1,2,3}
\newcommand{\set}[1]{\left\{ #1 \right\}}
%% We also redfine the negation symbol:
\renewcommand{\neg}{\sim}

\begin{document}

\subsection*{Solutions to homework 11:}

%%
%% There are 2 list environments, itemize and enumerate. They are almost identical, but each item in itemize is started with a bullet or dot, while each item in enumerate is numbered.
%%
\begin{enumerate}
%% This is where your actual homework will go.
\item Determine if the following sets are countable, and prove your answers.
\begin{itemize}
	\item (a) The set of all functions $f:\{0,1\} \rightarrow \mathbb{N}$.
	\begin{itemize}
	\item Let $G : S \rightarrow \mathbb{N} \times \mathbb{N}$ and be defined as $G(f) = \{ f(0) = a, f(1)=b:(a,b) \in \mathbb{N} \times \mathbb{N}\}.$
	\item Let $f,h \in S$ be functions such that $G(f) = G(h)$ then $G(f) = \{ f(0) = a, f(1)=b:(a,b) \in \mathbb{N} \times \mathbb{N}\} = \{ h(0) = a, h(1)=b:(a,b) \in \mathbb{N} \times \mathbb{N}\} = G(h).$
	\item This means that both $f,h$ have the same inputs for outputs $a,b$ hence $f=h$ due to the fact that the only possible inputs are 0 and 1.
	\item Which shows that $f$ is an injection.
	\item Now let $(c,d) \in \mathbb{N} \times \mathbb{N},$ and let's define a function $p$ so that $p(0) = c, p(1) = d$.
	\item Therefore by definition, $G(p) = (c,d)$ showing us that $p \in S$ this a surjection.
	\item Hence as $G$ forms a bijection with $|S| = |\mathbb{N} \times \mathbb{N}| = |\mathbb{N}|$ and $S$ is countable.
 	\end{itemize}
	\item (b) The set of all functions $f: \mathbb{N} \rightarrow \{0,1\}.$
	\begin{itemize}
	\item Let $G : S \rightarrow \mathcal{P}(\mathbb{N})$ and be defined as $G(f) = \{n:f(n) = 1\}$.
	\item Let $f,h \in S$ be functions such that $G(f) = G(h).$
	\item This implies that $G(f) = \{n:f(n) =1 \} = \{n : h(n) = 1\} = G(h).$
	\item This means that both $f,h$ have outputs 1 and 0 for the same inputs, therefore showing us that $f=h$.
	\item Therefore $G$ is an injection.
	\item Now let $X \in  \mathcal{P}(\mathbb{N})$ thus $X \subseteq \mathbb{N}$ and define a function $g : \mathbb{N} \rightarrow \{0,1\}$ as $g(x) = 1$ when $x\in X$ otherwise $g(x) = 0.$
	\item Hence by definition, $G(g)=X$ showing us that $G$ is a surjection.
	\item Therefore $G$ a bijection showing us that $|S| = |\mathcal{P}(\mathbb{N}) | = |\mathbb{R}|$. Thus $S$ is uncountable.
\end{itemize}
\end{itemize}
\item Prove the following statements
\begin{itemize}
	\item (a) If $A$ is countable but $B$ is uncountable, then $B-A$ is uncountable.
	\begin{itemize}
	\item Let's assume, to the contrary, that $B-A$ is countable.
	\item Then $(B-A) \cup A = (A\cup B)$ is countable.
	\item Therefore $B$ is countable, however this contradicts with out assumption.
	\item Hence, $B-A$ is uncountable.
	\end{itemize}
	\item (b) Between any real numbers $a,b$ such that $a<b$ there are uncountably many irrationals.
	\begin{itemize}
	\item We can see that we can represent this set of numbers as $S = \{x: a<x<b, x\in \mathbb{I}\}$.
	\item We can rephrase this expression to $S = \mathbb{R} - \{x:a<x<b, x\in \mathbb{Q}\}.$
	\item We know that the set $\mathbb{R}$ is uncountable and $\mathbb{Q}$ is countable.
	\item Therefore, any subset of $\mathbb{Q}$ is countable hence $\{x:a<x<b,x\in \mathbb{Q}\}$ is therefore countable.
	\item The original set $S = \mathbb{R} - \{x : a<x<b,x\in \mathbb{Q}\}$ is therefore uncountable. 
	\end{itemize}
\end{itemize}
\item Prove that $\mathbb{R}$ and $\mathbb{R}^{+} = \{x \in \mathbb{R} \mid x>0\}$ are equinumerous.
\begin{itemize}
\item Let's consider a function $f:\mathbb{R} \rightarrow \mathbb{R}^{+}$.
\item We define the function $f(x) = e^x$ for $x\in \mathbb{R}$.
\item Let $a,b \in \mathbb{R}$ such that $f(a)=f(b).$ Then 
\begin{align}
e^a &= e^b \\ e^a-e^b &=0 \\ e^b \cdot (e^{a-b} -1) &=0	
\end{align}
\item We know that $\forall n\in \mathbb{R},$ $e^n \geq 0.$ Hence $a-b=0$ showing that $a=b$.
\item Thus $f $ forms an injection and $|\mathbb{R}| \leq  |\mathbb{R}^{+}|$.
\item Now let's consider a function $g: \mathbb{R}^{+} \rightarrow \mathbb{R}$ and this function is defined as $g(x) =x$ for $x\in \mathbb{R}.$
\item Now let $c,d \in \mathbb{R}^{+}$ so that $g(c) =g(d)$.
\item Then, $g(c)=c=d=g(d)$ and thus shows us that $g$ forms an injection. 
\item Hence, $|\mathbb{R}^{+} | \leq |\mathbb{R}|.$ And as we proved from earlier that $|\mathbb{R}| \leq |\mathbb{R}^{+}|$ by CSB $\mathbb{R}$ and $\mathbb{R}^{+}$ are equinumerous.
\end{itemize}

\item Let $S,T$ be sets. Prove the following
\begin{itemize}
	\item (a) If $|S| \leq |T|$ then $|\mathcal{P}(S)| \leq |\mathcal{P}(T)|.$
	\begin{itemize}
	\item Let $f : S \rightarrow T$ be a well-defined function  that is an injection.
	\item Now let $g:\mathcal{P}(S) \rightarrow \mathcal{P}(T)$ be defined as $g(f)=\{f(A) =B : A \subseteq S, B \subseteq T\}.$
	\item Let $P,Q \in \mathcal{P}(S), X \in \mathcal{P}(T)$ such that $g(P) = g(Q) = X.$
	\item Then this implies that $g(P) = \{f(P)= X:P \subseteq S, X \subseteq T\} = \{f(Q) = X:Q\subseteq S, X \subseteq T\}=g(Q).$
	\item Hence by definition $P=Q$ showing that $g$ is an injection as we know that $f$ is also an injection therefore $|\mathcal{P}(S)| \leq |\mathcal{P}(T)|.$
	\end{itemize}
	\item (b) If $|S| = |T|$ then $|\mathcal{P}(S)| = |\mathcal{P}(T)|.$
	\begin{itemize}
	\item Let $h:S \rightarrow T$ form a bijection.
	\item Now let $g:\mathcal{P}(S) \rightarrow \mathcal{P}(T)$ be defined as $g(h)=\{h(A) =B : A \subseteq S, B \subseteq T\}.$
	\item As we proved in (a) we know that $g$ forms an injection.
	\item Now let's prove that $g$ forms a surjection.
	\item Let $X \subseteq S, Y\subseteq T$ such that $f(X) =Y$.
	\item Then by definition $g(f) = \{f(X) = Y: X \subseteq S, Y \subseteq T\}$.
	\item Hence is a surjection as required.
	\item Therefore, $g$ is a bijection hence $|\mathcal{P}(S)| = |\mathcal{P}(T)|.$.
	\end{itemize}
\end{itemize}
\item Show that there exist infinitely many pairs of distinct natural numbers $a,b$ such that $17^a-17^b$ is divisible by 2022.
\begin{itemize}
	\item Consider a sequence of 2023 numbers $17^1,17^2,17^3,\cdots,17^{2023}$.
	\item There are at most 2022 remainders when divided by 2022.
	\item However there exists 2023 numbers in the sequence.
	\item Thus there must exists two numbers with the same remainder when divided by 2022. 
	\item Then $\exists a,b \in \mathbb{N}$ where $17^a = 2022k+r,17^b=2022\ell + r$ where $r \in \mathbb{N}$ and $k,\ell \in \mathbb{Z} , k,\ell \geq 0.$
	\begin{align}
		17^b-17^a &= (2022\ell + r) - (2022k+r) \\&=2022(\ell-k)
	\end{align}
	\item Hence, $2022 \mid (17^b-17^a).$
	\item In the previous example, we considered the sequence with the interval $[17^1,17^{2023}]$.
	\item However this relation will still be satisfied for other intervals too.
	\item To generalize, there will always exist two natural numbers $a,b$ where $2022 \mid (17^b-17^a)$ for all intervals $[17^n,17^{n+2022}],n\in \mathbb{N}.$
	\item We know that $\mathbb{N}$ is an infinite set hence there are infinitely many pairs that exist.
\end{itemize}
\item Prove that $(-\infty,-\sqrt{29})$ and $\mathbb{R}$ are equinumerous by constructing an explicit bijection.
\begin{itemize}
	\item Let $f:(-\infty,-\sqrt{29}) \rightarrow \mathbb{R}$ be defined as $f(x) = \log(-x-\sqrt{29})$
	\item And also let $g:\mathbb{R} \rightarrow(-\infty,-\sqrt{29})$ be defined as $g(y)=-e^y-\sqrt{29}.$
	\item Then 
	\begin{align}
		f(g(y)) &= \log(-(-e^y-\sqrt{29})-\sqrt{29}) \\ &=\log(e^y) \\&=y \\ g(f(x))&=-e^{\log(-x-\sqrt{29})}-\sqrt{29} \\&=-(-x-\sqrt{29}) -\sqrt{29} \\&=x
	\end{align}
	\item We can see that $f\circ g$ and $g \circ f$ are both identity functions.
	\item Thus $f$ has a two side inverse which is $g$.
	\item Therefore $|(-\infty,-\sqrt{29})| = |\mathbb{R}|$. Thus equinumerous as required.
\end{itemize}
\item Prove or disprove: for any non-empty sets $A,B,C$ if $| A \times B| = |A \times C|$ then $|B|  = |C| $.
\begin{itemize}
	\item First, let $f:|A \times B| \rightarrow |A \times C|$ be defined as $f((a_k,b_n))=\{(a_k,c_n) : k,n \in \mathbb{N}\}$.
	\item We know that $|A\times B|=|A \times C|$ hence we can see that $f$ is a bijection.
	\item Now let $g:B \rightarrow C$ be defined as $g(b_k)=\{c_k : f((a_1,b_k)) =(a_1,c_k)\}.$
	\item Let's prove whether $g$ is a bijection.
	\item Let $b_i,b_j$ such that $g(b_i)=g(b_j)$,  then $g(b_i) = \{c_i : f((a_1,b_i))=(a_1,c_i)\} = \{c_j : f((a_1,b_j)) = (a_1,c_j)\} = g(b_j).$
	\item We know that $f$ is a bijection hence $c_i = c_j$, therefore showing us that $g$ is an injection.
	\item Now let $c_\ell \in C$, since $f$ is a surjection $\exists (a_1,b_\ell) \in A \times B$ such that $f((a_1,b_\ell)) = (a_1,c_\ell)$.
	\item Therefore, by the definition of $g$ $\exists b_k \in B$ where $g(b_k)=c_k$.
	\item Thus, $g$ is also a surjection showing us that $g$ is a bijection.
	\item Since we proved that $g$ is a bijection, $|B|=|C|.$
\end{itemize}
\item Let $A$ be a finite set and $f: \mathbb{R} \rightarrow A.$ Show that there exists some $a \in A$ such that $f^{-1}(\{a\})$ is uncountable.
\begin{itemize}
	\item Let's assume, to the contrary, that $f^{-1}(\{a\})$ is countable.
	\item Then we $\mathbb{R}$ can be expressed with $f^{-1}(\{a\})$ as
	\begin{center}
	$\mathbb{R} = \bigcup_{a \in A} f^{-1}(\{a\}), \forall a \in A$.	
	\end{center}
	\item As we know that $A$ is a finite set, then $\mathbb{R}$ is the union of a finite amount of sets, therefore shows us that $\mathbb{R}$ is finite.
	\item However, this contradicts with the fact that $\mathbb{R}$ is an infinite set.
	\item Hence, by contradiction we can declare that $f^{-1}(\{a\})$ is uncountable.

\end{itemize}
\end{enumerate}




%% Anything that comes after the ``\end{document}'' will be ignored, not just by us but by the latex editor too.
\end{document}

See, we can have stuff here which will not appear in the compiled file.